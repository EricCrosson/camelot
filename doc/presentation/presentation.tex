\documentclass[t, pdftex]{beamer}  
%Use Cockrell School Theme.  Optional department name.  Must %ecape, i.e. use 
%backslash, to preserve spaces.  The default is ``Cockrell School of Engineering''
%\usetheme[]{Cockrell}                 
%\usetheme[dept=Aerospace\ Engineering\ and\ Engineering\ Mechanics]{cockrell}                 
%\usetheme[dept=Biomedical\ Engineering]{cockrell}                 
%\usetheme[dept=Chemical\ Engineering]{cockrell}                 
%\usetheme[dept=Civil,\ Architectural\ and\ Environmental\ Engineering]{cockrell}
\usetheme[dept=Electrical\ and\ Computer\ Engineering]{Cockrell}                 
%\usetheme[dept=Mechanical\ Engineering]{cockrell}                 
%\usetheme[dept=Materials\ Science\ and\ Engineering]{cockrell}                 
%\usetheme[dept=Petroleum\ and\ Geosystems\ Engineering]{cockrell}                 

% Add preamble packages here
%\usepackage{etex}
%\usepackage[bigfiles]{media9}
\usepackage{graphicx}
\graphicspath{{../figs/}}
\usepackage{caption}

\usepackage{listings,multicol}
\definecolor{ForestGreen}{rgb}{0.13,0.55,0.13}
\lstset{
    language         = Promela,
    basicstyle       = \ttfamily,
    keywordstyle     = \bfseries\color{blue},
    stringstyle      = \color{magenta},
    commentstyle     = \color{ForestGreen},
    showstringspaces = false,
}

%Enable cancelto in math
\usepackage{cancel}
\renewcommand{\CancelColor}{\color{utorange}}

%Add bibliography file location for citiation
% \bibliography{example.bib}


\title{Verifying Distributed Algorithms}
\subtitle{in Promela}
\author{Eric Crosson \\ Nhan Do \\ Stormy Mauldin \\ Daniel Officewala}
\institute{EE 360P}
\date{\today}


\begin{document}

%Creates title frame from title, subtitle, author, institute, and date above
\titleframe

%Supports table of contents
\frame{\frametitle{Outline}\tableofcontents}

%Section commands will define what's shown in TOC

%First frame
\section{Promela Overview}
\begin{frame}
    \frametitle{Promela Overview}
    \begin{itemize}
      \item Promela is \textbf{Pr}ocess \textbf{Me}ta \textbf{La}nguage
      \item Spin compiles Promela into C
      \item C executables assert system invariants at each simulated state
    \end{itemize}
\end{frame}

\section{Dining Philosophers}
\begin{frame}[c]
  \frametitle{Dining Philosophers}
  \begin{multicols}{2}
    \textbf{Most Philosophers}
    \begin{itemize}
      \item Want to eat:
      \begin{itemize}
        \item Get left fork (or wait)
        \item Get right fork (or wait)
      \end{itemize}
      \item After eating:
      \begin{itemize}
        \item Release both forks
        \item Contemplate life until hungry
      \end{itemize}
    \end{itemize}
    \columnbreak
    \textbf{One "Special" Philosopher}
    \begin{itemize}
      \item Want to eat:
      \begin{itemize}
        \item Get left fork (or wait)
        \item Get right fork (if fail, release \textit{both} forks)
      \end{itemize}
      \item After eating:
      \begin{itemize}
        \item Release both forks
        \item Contemplate life until hungry
      \end{itemize}
    \end{itemize}
  \end{multicols}
\end{frame}

\begin{frame}[c]
  \frametitle{Dining Philosopher Analysis}
  \begin{multicols}{2}
    \begin{itemize}
      \item \textbf{Mutually Exclusive}
      \begin{itemize}
        \item A philospher must acquire both shared forks before eating
      \end{itemize}
      \item \textbf{Deadlock}
      \begin{itemize}
        \item The one special philosopher will always surrender both forks, allowing someone else to start eating.
      \end{itemize}
      \item \textbf{Starvation}
      \begin{itemize}
        \item The special philosopher always surrenders forks in case of conflict. May never get a change to eat.
      \end{itemize}
    \end{itemize}
    \columnbreak
    \begin{figure}
      \begin{minipage}{.5\textwidth}
        \centering
        \includegraphics[scale=.20]{DiningPhilosophers}
        \captionof{figure}{Hungry, hungry philosophers}
      \end{minipage}
    \end{figure}
  \end{multicols}
\end{frame}



\section{Token Ring}
\begin{frame}[c]
    \frametitle{Token Ring Algorithm}
    % \TeX\ \cite{knuth1989} and \LaTeX\ \cite{lamport1986} are great for equations
    \begin{itemize}
		\item Simple and easily scalable
		 \begin{itemize}
          \item Pass token around ring of processes
          \item Only processes with token can enter CS
		  \item No Starvation
        \end{itemize}
    \end{itemize}
	
	
 \begin{figure}
	 \begin{minipage}{.5\textwidth}
      \centering
      \includegraphics[scale=.30]{tokenring}
      \captionof{figure}{Token Ring Algorithm}
      \label{fig:Token Ring}
    \end{minipage}
\end{figure}
	
\end{frame}

\begin{frame}
	\frametitle{Token Ring Implementation}
	  \begin{multicols*}{3}
    \lstinputlisting[language=Promela, basicstyle=\tiny]{../figs/tokenring.pml}
  \end{multicols*}
\end{frame}



\section{Lamport's}
\begin{frame}[c]
    \frametitle{Lamport's}
    % \TeX\ \cite{knuth1989} and \LaTeX\ \cite{lamport1986} are great for equations
    \begin{block}{An equation block}
        \[ \vec{F} = m \vec{a} \]
    \end{block}
    \begin{itemize}
        \item Second instance of citations use short citation hyperlinked to original.
    \end{itemize}
\end{frame}

\section{Szymanski's}
\begin{frame}[c]
    \frametitle{Szymanski's Algorithm}
    \begin{itemize}
      \item Extension of Lamport's
        \begin{itemize}
          \item satisfies linear wait
          \item three booleans per process
        \end{itemize}
      \item Extension of Lamport's
    \end{itemize}
  % PRO POINTS: make your own charts here, don't be a scrub
  \begin{figure}
    \centering
    \begin{minipage}{.5\textwidth}
      \centering
      \includegraphics[scale=.3]{szymanskis_carosel}
      \captionof{figure}{Szymanski's Algorithm}
      \label{fig:szymanskis_carosel}
    \end{minipage}%
    \begin{minipage}{.5\textwidth}
      \centering
      \includegraphics[scale=.61]{szymanskis_booleans}
      \captionof{figure}{State-tracking booleans}
      \label{fig:szymanskis_booleans}
    \end{minipage}
  \end{figure}
\end{frame}

\begin{frame}
  \frametitle{Szymanski's Implementation}
  \begin{multicols*}{3}
    \lstinputlisting[language=Promela, basicstyle=\tiny]{../figs/szymanski.pml}
  \end{multicols*}
\end{frame}


\section{Questions}
\begin{frame}[c]
    \frametitle{Questions}  % verbally, say ``What are your questions?'' 
    % \TeX\ \cite{knuth1989} and \LaTeX\ \cite{lamport1986} are great for equations
    Thank you for your time.
\end{frame}


% This command is causing problems
% I am not sure what we're missing out on by not having it.
%\lastframe%
\end{document}
