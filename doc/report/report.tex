\documentclass[12pt]{article}
%
%Margin - 1 inch on all sides
%
\usepackage[letterpaper]{geometry}
\geometry{top=1.0in, bottom=1.0in, left=1.0in, right=1.0in}

%
%Doublespacing
%
\usepackage{setspace}
\doublespacing
%
%Babel package for multiple language typesetting
%
%\usepackage[english,german]{babel}
%\usepackage[T1]{fontenc}
%\usepackage[latin1]{inputenc}
%
%Setting the font
%
\usepackage{times}
%
%Rotating tables (e.g. sideways when too long)
%
\usepackage{rotating}
%
%For multiple rows in tables
%
\usepackage{multirow}
%
%Line numbering in verse environment
%
\usepackage{lineno}

%
%Fancy-header package to modify header/page numbering (insert last name)
%
\usepackage{fancyhdr}
\usepackage{graphicx}
\graphicspath{ {../img/} }
\pagestyle{fancy}
\lhead{}
\chead{}
\rhead{\thepage}
\lfoot{}
\cfoot{}
\rfoot{}
\renewcommand{\headrulewidth}{0pt}
\renewcommand{\footrulewidth}{0pt}
%To make sure we actually have header 0.5in away from top edge
%12pt is one-sixth of an inch. Subtract this from 0.5in to get headsep value
\setlength\headsep{0.333in}
%
%Works cited environment
%(to start, use \begin{workscited...}, each entry preceded by \bibent)
% - from Ryan Alcock's MLA style file
%
\newcommand{\bibent}{\noindent \hangindent 40pt}
\newenvironment{workscited}{\newpage \begin{center} Works Cited \end{center}}{\newpage }

%
%Begin document
%
\begin{document}
\begin{flushleft}
%%%%First page name, class, etc
Eric Crosson, Nhan Do, William Mauldin, and Daniel Officewala \\
Professor Vijay Garg\\
EE 360P \\
\today \\

%%%%Title
\begin{center}
  \emph{\textbf{Abstract}: This report evaluates the use of Spin to verify
    Promela models of distributed systems. It was written as the final project
    for Professor Vijay Garg's Concurrent and Distributed Computing course at
    the University of Texas at Austin.}
\end{center}

%%%%Changes paragraph indentation to 0.5in
\setlength{\parindent}{0.5in}
%%%%Begin body of paper here
\section{Introduction}
Spin is a multi-threaded software verification tool that supports the high-level
language Promela to specify models of systems [1]. Given a model, Spin either
performs random simulations of the system's execution or generates a C program
that verifies the system's correctness properties, including deadlocks,
unspecified receptions, and unexecutable blocks, as well as system invariants
and other properties defined by the user. Promela allows the user to define
processes, message channels, and variables.

\section{Project Description}
The intent of this project is to verify four models of distributed algorithms
written in Promela. We chose to model the Dining Philosophers Algorithm [2], the
Token Ring Algorithm [3], Szymanski's Mutual Exclusion Algorithm [4], and Chandy
and Lamport's Global Snapshot Algorithm [5]. We based our implementation heavily
upon existing implementations of these algorithms, modified to suit our specific
uses.

Each of the four team members was responsible for researching one algorithm,
finding a suitable model, and re-implementing it for use within our directory
structure and build system. Eric created the majority of the build system and
set up Docker [6] to allow for our research to be reproducible, all of which is
available on Github [7]. William assisted with this process by helping to write
some of the CMake files necessary to build the project. Eric focused his
research on Szymanski's Algorithm, Daniel on the Token Ring Algorithm, William
on Chandy and Lamport's Algorithm, and Nhan on the Dining Philosophers
Algorithm.

\section{Performance Results}
We were able to successfully build all of the algorithms and run them in Spin,
but only three out of the four passed the verification tests that we used. We
modified all of the existing Promela models such that any defined constants are
now able to be passed as parameters and that any assertions are extracted to
separate test files.

\subsection{Dining Philosophers Algorithm}
The Dining Philosophers Algorithm is a widely used algorithm for guaranteeing
mutual exclusion between multiple processes sharing resources. We used a model
created by [INSERT MODEL MAKER HERE][8] for verification. We used one test file
for this model to ensure deadlock resolution. (EXPLAIN MORE HERE). When run in
Spin, we had no assertion failures, so deadlock was verified as
impossible. However, we determined that the algorithm was not free from
starvation, since it is theoretically possible for the special process (the
process that surrenders a held resource when a second resource is unavailable)
to constantly surrender its held resource if another, faster process constantly
gets access to the second resource prior to the special process. (SHOW RESULTS
BELOW)

\subsection{Token Ring Algorithm}
In the Token Ring Algorithm, a ring of processes is constructed, and each
process is assigned a specific position in the ring. A token for critical
section (CS) access is passed around the ring, and a process can either use it
to access the CS or pass it along. We used a model created by [INSERT MODEL
MAKER HERE][9] in order to model the algorithm, and a set of mutual exclusion
test cases created by Fumiyoshi Kobayashi [10]. As expected, when we ran the
tests against the model in Spin, safet and liveness were confirmed, since a
process is only allowed to enter the CS if it holds the sole token, and the
token is continually passed either immediately or after exiting the CS. However,
these tests do not account for faults (lost token). In order to implement
fault-tolerance, the algorithm could be modified by having processes send an
“acknowledgement” upon receiving a token from a previous process.

\subsection{Szymanski's Mutual Exclusion Algorithm}
insert text here

\subsection{Chandy and Lamport's Global Snapshot Algorithm}
Chandy and Lamport's Global Snapshot Algorithm defines a method to take a global
snapshot of a distributed system, which has proven to be a challenge due to the
absence of shared memory and a shared clock. In this algorithm, the global
snapshot is determined by collecting all process states (messages recorded by a
given process prior to receiving a marker) and channel states (messages recorded
by a processes after receiving a marker). All channels are assumed to be
unidirectional and FIFO. The algorithm theoretically guarantees consistency (all
recorded states are concurrent in the happened-before model). To test
consistency, we used a model created by Mordecai Ben-Ari [11] that includes a
sender process and a receiver process communicating over a channel. The two
tests we extracted theoretically should have confirmed that $i)$ the last
message sent was always equal to the message at the marker (correct channel
states), and $ii)$ any message recorded was always less than the message at the
marker (correct process states). Unfortunately, due to our previous
unfamiliarity with the Promela language, both of our extracted tests continually
resulted in assertion failures. We attribute this to our lack of experience with
Promela due to the fact that this model has previously been verified in its
original state without separate test files and parameter passing.

\section{Conclusions}
Although not all of our Promela models were able to pass their respective
verification tests, we were able to build and execute them using Spin. Given
more time to learn the grammar and syntax of Promela, we believe that we could
devise more robust tests that produce more accurate results.

\section{References}
[ 1] http://spinroot.com/spin/Man/Manual.html \newline
[ 2] DINING PHILOSOPHER INFORMATION SOURCE \newline
[ 3] TOKEN RING INFORMATION SOURCE \newline
[ 4] SZYMANSKI INFORMATION SOURCE \newline
[ 5] CHANDY INFORMATION SOURCE (STORMY FIRST CITATION) \newline
[ 6] https://www.docker.com/\newline
[ 7] https://github.com/stormosson/camelot\newline
[ 8] LINK TO DINING PHILOSOPHER MODEL
[ 9] LINK TO TOKEN RING MODEL \newline
[10] www.ueda.info.waseda.ac.jp/~kobayashi/Promela/benchmark/index.html\newline
[11] CHANDY MODEL LINK (STORMY SECOND CITATION [i think])
\end{flushleft}
\end{document}
\}
